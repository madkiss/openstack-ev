\documentclass[draft]{scrartcl}

\usepackage[utf8]{inputenc}
\usepackage[german]{babel}

\usepackage{scrjura}
\usepackage{paralist}
\usepackage{eurosym}
\usepackage{totpages}
\usepackage{everyshi}
\usepackage{keyval}

%\usepackage{draftwatermark}
%\SetWatermarkText{Entwurf 2014-05-18}
%\SetWatermarkScale{3}
%\SetWatermarkColor[gray]{0.95}

\usepackage{totpages}
\usepackage{scrpage2}
\usepackage{titling}

\usepackage{geometry}
\setlength{\parindent}{0pt}
\setlength{\parskip}{0.5em}

\frenchspacing

\setkeys{contract}{preskip=2\parskip, postskip=\parskip}

\KOMAoption{ref}{short,nosentence}

\addtokomafont{title}{\rmfamily}
\addtokomafont{section}{\rmfamily}
\addtokomafont{contract.Paragraph}{\rmfamily}

\newcommand{\VereinsName}{OpenStack DACH}

\begin{document}

\selectlanguage{german}

\title{Entwurf einer Satzung des Vereins "`\VereinsName\ [e.V.]"'}
\date{}
\author{}
\def\fassungsdatum{26.\ November 2014}

\begin{center}
  \LARGE \thetitle
\end{center}

\begin{flushright}
  Fassung vom \fassungsdatum
\end{flushright}

\pagestyle{scrheadings}
\clearscrheadfoot
\ofoot{\footnotesize Seite \thepage\ von \pageref{TotPages}}

{\footnotesize Diese Vereinssatzung wurde am ??.\ ??.\ 2014 in der Versammlung
  zur Gründung des Vereins "`\VereinsName\ [e.V.]"\ offiziell errichtet. Sie
  umfasst \pageref{TotPages} numerierte Seiten, inklusive dieser.}

\bigskip

\section*{Präambel}\enlargethispage{\textheight}
\small

OpenStack ist ein universelles Betriebssystem für Cloud-Computing-Umgebungen
und mittlerweile das größte Projekt seiner Art. Entwickler in aller Welt
beteiligen sich aktiv an der Weiterentwicklung der Software, die pro Jahr
zwei neue Versionen veröffentlicht. Der Fokus der Umgebung liegt dabei ganz
eindeutig auf Skalierbarkeit: OpenStack gibt Admins Werkzeuge an die Hand,
die den Aufbau und den Betrieb großer Computing-Umgebungen vereinfachen
und bequem werden lassen.

Mittlerweile hat OpenStack weltweit eine große Fan-Gemeinde. Das Projekt hat
insbesondere dem "`User Group"'-Prinzip neues Leben eingehaucht, indem es die
OpenStack-Entwickler und die Administratoren von OpenStack-Clouds regelmäßig
bei "`Meetups"' zusammen bringt. Treffen dieser Art bieten die Möglichkeit
zum Austausch über verschiedene Themen, die mit OpenStack zu tun haben.

Neben dem größten Event dieser Art, dem halbjährlich stattfindenden OpenStack
Summit, organisieren Mitglieder der OpenStack-Gemeinschaft regelmäßig auch
regionale Events überall in der Welt. Die lokalen Meetups bieten vor Ort
die Möglichkeit, sich über aktuellen Entwicklungen rund um OpenStack zu
informieren, eigene Entwicklungen und Produkte vorzustellen und mit anderen
OpenStack-Begeisterten Wissen auszutauschen. Für den deutschen Sprachraum
(DACH) haben bisher außerdem zwei überregionale Events stattgefunden, die
als "`OpenStack DACH Tag"' jeweils zusammen mit dem LinuxTag in Berlin in
den Jahren 2013 und 2014 stattgefunden haben.

Die Möglichkeit, die Veranstaltung zusammen mit dem LinuxTag durchzuführen,
entfällt allerdings im Jahre 2015, weil der LinuxTag nicht wie gewohnt
stattfindet. Viele Mitglieder der deutschen OpenStack-Gemeinschaft sind
sich jedoch darin einig, dass es auch 2015 wieder eine entsprechende
Veranstaltung in der DACH-Region rund um OpenStack geben sollte. Sie
gründen deshalb den \VereinsName\ e.V., der als juristische Person auf der
einen Seite als Organisator eines solchen Events (und ggf. weiterer
lokaler Events) mit allen Rechten und Pflichten auftreten kann, der aber
andererseits auch in der Lage ist, Spenden für die Durchführung eben jener
Veranstaltungen zu sammeln.

Als Verein ist \VereinsName\ e.V. auch an demokratische Prinzipien gebunden
und unabhängig von einzelnen Unternehmen. Das stellt sicher, dass etwaige
Veranstaltungen wie der "`OpenStack DACH Tag"' eine offene Plattform für alle
interessierten Unternehmen und Organisationen bleiben, statt zur "`Hausmesse"'
einzelner Firmen zu werden. Die Gründungsmitglieder von \VereinsName\ e.V.
halten ausdrücklich fest, dass dies für sie eine der Hauptmotivationen für die
Gründung des Vereins darstellt.

Dieser Satzung liegt die Satzung des "`DebConf e.V."' zugrunde, die unter
den Bedingungen der CC-by-SA-Lizenz 4.0 von Martin F. Krafft veröffentlicht
worden ist.

\normalsize

\pagebreak
\ihead{\footnotesize \thetitle}
\ohead{\footnotesize Fassung vom \fassungsdatum}
\begin{contract}

\Paragraph{title={Name, Sitz, Gerichtsstand und Geschäftsjahr}}

Der Verein führt den Namen "`\VereinsName"'; nach der beabsichtigten
Eintragung in das Vereinsregister mit dem Zusatz "`e.V."' (fortan: der
Verein).

Der Verein hat seinen Sitz in Berlin.

Gerichtsstand ist Berlin.

Das Geschäftsjahr ist das Kalenderjahr. Der Zeitraum von der Gründung des
Vereins bis zum Jahresende bildet ein Rumpfgeschäftsjahr.

\Paragraph{title=Zweck}

\parnumberfalse

\label{Zweck}
Zweck des Vereins ist die Förderung von Volksbildung. Insbesondere geht es
dem Verein darum, durch die Unterstützung entsprechender Veranstaltungen
bei der Vermittlung von Wissen rund um das Thema OpenStack aktiv zu sein.

Der Zweck des Vereins wird insbesondere durch die Organisation von öffentlich
zu\-gäng\-li\-chen Veranstaltungen verwirklicht, die OpenStack selbst oder
mit OpenStack in Verbindung stehende Projekte der Allgemeinheit näher
bringen und somit der Volksbildung dienen, wie z.B.

\begin{compactitem}
  \item Konferenzen,
  \item Vorträge und Workshops,
  \item Lehrgänge und andere Weiterbildungsangebote.
\end{compactitem}

Der Verein erhebt für seine Veranstaltungen grundsätzlich keine Teilnahmegebühr.
Es steht ihm jedoch frei, für einzelne Veranstaltungen einen adäquaten Beitrag zur
Abdeckung von Unkosten von den Teilnehmern seiner Veranstaltungen einzuheben.

Wenn technisch möglich stellt der Verein Video-Materialien seiner Veranstaltungen
öf\-fent\-lich und kostenfrei zur Verfügung, um den wissenschaftlichen wie kulturellen
Mehrwert auch über die Veranstaltungen hinaus zu gewährleisten.

\Paragraph{title=Gemeinnützigkeit}\label{Gemeinnuetzigkeit}

Der Verein verfolgt ausschließlich und unmittelbar gemeinnützige Zwecke im
Sinne des Abschnitts "`Steuerbegünstigte Zwecke"' der Abgabenordnung.

Der Verein ist selbstlos tätig; er ist konfessionell und parteipolitisch
neutral. Er verfolgt nicht in erster Linie eigenwirtschaftliche Zwecke.

Dem Verein stehen folgende Mittel zur Verfügung:

\begin{compactitem}
  \item Beiträge der Mitglieder,
  \item Zuwendungen und Schenkungen,
  \item Vermögen und
  \item seine Erträge aus Ergebnissen der Vereinsarbeit.
\end{compactitem}

Die Mittel des Vereins werden ausschließlich und unmittelbar zu den
satzungsgemäßen Zwecken verwendet. Die Mitglieder erhalten als solche
keinerlei Zuwendung aus Mitteln des Vereins. Zweckmäßiger Spesenersatz stellt
keine Zuwendung dar. Niemand darf durch Ausgaben, die dem Zwecke des Vereins
fremd sind, oder durch unverhältnismäßig hohe Vergütungen, begünstigt werden.

Um den Erkenntnisgewinn bei den Teilnehmern von Veranstaltungen zu erhöhen,
lädt der Verein namhafte, verdienstvolle und bedürftige Teilnehmer ein und
übernimmt die hierfür anfallenden Reise-, Unterbringungs- und
Verpflegungskosten.

Bei Ausscheiden eines Mitgliedes aus dem Verein oder bei Vereinsauf\/lösung
erfolgt keine Rückerstattung etwaig eingebrachter Vermögenswerte.

Eine Änderung des Vereinszwecks darf nur innerhalb des von \ref{Zweck}
gegebenen Rahmens erfolgen.

Bei der Auf\/lösung des Vereins oder beim Wegfall des steuerbegünstigten
Zwecks fällt das Vereinsvermögen an eine juristische Person des öf\-fent\-lichen
Rechts oder eine andere steuerbegünstigete Körperschaft zwecks Verwendung
für die Förderung der Bildung im Sinne dieser Satzung.

\Paragraph{title=Mitgliedschaft}

Es gibt aktive Mitglieder, Fördermitglieder, sowie Ehrenmitglieder.

Aktive Mitglieder des Vereins können ausschließlich natürliche Personen
werden.

Juristische Personen, Handelsgesellschaften, nicht rechtsfähige Vereine,
Anstalten und Kör\-pers\-chaften des öff\-ent\-lich\-en Rechts und andere
nicht-natürliche Personen oder Vereinigungen können nur als Fördermitglieder
aufgenommen werden. Fördermitglieder haben keine Stimmrechte.

Die Beitrittserklärung erfolgt in Textform gegenüber dem Vorstand. Über die
Annahme der Beitrittserklärung entscheidet der Vorstand. Die Mitgliedschaft
beginnt mit der Annahme der Beitrittserklärung.

Die Mitgliederversammlung kann solche Personen, die sich besondere Verdienste
um den Verein oder um die von ihm verfolgten satzungsgemäßen Zwecke erworben
haben, zu Ehrenmitgliedern ernennen. Ehrenmitglieder haben alle Rechte eines
aktiven Mitglieds. Sie sind von Beitragsleistungen befreit.

\SubParagraph{title=Jahresbeitrag}

Der Verein erhebt einen Jahresbeitrag, der durch die Mitgliederversammlung
per Beitragsordnung festgesetzt wird und für das jeweilige Kalenderjahr in
voller Höhe im Voraus zu zahlen ist. Bei einem Eintritt im Laufe eines
Kalenderjahres ist der Jahresbeitrag in voller Höhe zu entrichten.

Im begründeten Einzelfall kann für ein Mitglied durch Vorstandsbeschluß ein
von der Beitragsordnung abweichender, niedrigerer Beitrag festgesetzt werden.

\SubParagraph{title=Rechte und Pflichten}

Die Stimmrechte aktiver Mitglieder (inkl.\ Ehrenmitglieder) werden gemäß
\refParagraph{Stimmrechte}~\refParS{Stimmrechte} ausgeübt. Ansonsten bestehen
keine Mehransprüche auf Vereinsleistungen gegenüber der Öffentlichkeit.

Die Mitglieder sind verpflichtet, die satzungsgemäßen Zwecke des Vereins zu
unterstützen und zu fördern. Sie sind ebenso verpflichtet, die in der von der
Mitgliederversammlung festgelegten Beitragsordnung festgesetzten Beiträge zu
zahlen.

Jedes einzelne Mitglied hat Sorge dafür zu tragen, dass dem Vorstand zu jeder
Zeit eine zustellungsfähige Adresse des Mitglieds bekannt ist.

\SubParagraph{title=Beendigung der Mitgliedschaft}

Die Mitgliedschaft endet durch

\begin{compactitem}
  \item die Austrittserklärung des Mitglieds unter Einhaltung einer Frist von
  drei Monaten zum Ende des Kalenderjahres, wobei der Austritt dem Vorstand in
  Textform (z.B. mittels E-Mail, Brief oder Fax) erklärt werden muss; den Nachweis
  über den Eingang der Kündigung beim Vorstand führt der Absender.
  \item den Tod von natürlichen Personen.
  \item die Auf\/lösung und Erlöschung von juristischen Personen, Handelsgesellschaften,
  nicht rechtsfähigen Vereinen sowie Anstalten und Körperschaften des öf\-fent\-lichen
  Rechts.
  \item Ausschluss.
\end{compactitem}

Die Beitragspflicht für das laufende Geschäftsjahr bleibt hiervon unberührt.

Der Vorstand kann die Mitgliedschaft eines Mitglieds für erloschen erklären, wenn

\begin{compactitem}
  \item das Mitglied den per Beitragsordnung festgesetzten Jahresbeitrag
  drei Monate nach Beginn des Geschäftsjahres noch nicht gezahlt hat und
  \item trotz Mahnung durch eingeschriebenen Brief, der den Hinweis auf
  das Erlöschen der Mitgliedschaft enthalten muss,
  \item der fällige Betrag nicht innerhalb eines Monats beim Verein
  eingeht.
\end{compactitem}

Diese Regelung gilt sinngemäß auch für neue Vereinsmitglieder, wobei die
dreimonatige Frist hier mit dem Tag der Annahme der Beitrittserklärung
durch den Vorstand beginnt.

Erlischt die Mitgliedschaft eines Mitglieds, so hebt dies nicht die
Verpflichtung zur Zahlung fälliger Beiträge auf.

\SubParagraph{title=Ausschluss eines Mitglieds}

Ein Mitglied kann durch Beschluss des Vorstandes ausgeschlossen werden, wenn es
das Ansehen des Vereins schädigt, die Satzung des Vereins vorsätzlich
verletzt oder wenn ein sonstiger wichtiger Grund vorliegt. Der Vorstand muß
dem auszuschließenden Mitglied den Beschluss in Textform unter Angabe von Gründen
mitteilen.

Gegen seinen Ausschluss kann das Mitglied mit aufschiebender Wirkung die nächste
Mitgliederversammlung anrufen, die eine endgültige Entscheidung trifft.

Bei einer Abstimmung über den Beschluss bzw.\ den Ausschluss hat das betroffene
Mitglied keine Stimme. Sollen mehrere Mitglieder ausgeschlossen werden, so ist
über jedes einzeln abzustimmen.

\Paragraph{title=Organe des Vereins}

Die Organe des Vereins sind

\begin{compactenum}[\hspace{2em}1.]
  \item die Mitgliederversammlung,
  \item der Vorstand,
  \item die Ressorts.
\end{compactenum}

\Paragraph{title=Mitgliederversammlung}\label{Mitgliederversammlung}

Oberstes Beschlussorgan ist die Mitgliederversammlung. Ihrer Beschlussfassung
unterliegen:

\begin{compactenum}[\hspace{2em}1.]
  \item die Genehmigung des vom Vorstand aufgestellten Haushaltsplanes für das
    näch\-ste Geschäftsjahr,
  \item die Entgegennahme des Finanzberichtes,
  \item die Entlastung des Vorstandes,
  \item die Wahl und Abberufung der einzelnen Vorstandsmitglieder,
  \item die Bestellung von Finanzprüfern, sowie Entgegennahme der
    Prüfungsberichte,
  \item Satzungsänderungen,
  \item die Festlegung der Beitragsordnung,
  \item die Richtlinie über die Erstattung von Reisekosten und Auslagen,
  \item Anträge des Vorstandes und der Mitglieder,
  \item Entscheidungen über Beschwerden gegen Ablehnungen von
    Aufnahmeanträgen,
  \item die Ernennung von Ehrenmitgliedern,
  \item Ausschluss eines Vereinsmitglieds,
  \item die Auf\/lösung des Vereins.
\end{compactenum}

Die ordentliche Mitgliederversammlung findet jedes Jahr statt. Außerordentliche
Mitgliederversammlungen werden auf Beschluss des Vorstandes abgehalten, wenn die
Interessen des Vereins dies erfordern, oder wenn mindestens ein Zehntel der
Mitglieder dies unter Angabe des Zwecks in Textform beantragt.

Die Mitgliederversammlung kann sich eine Geschäftsordnung geben.

\SubParagraph{title=Einberufung und Tagesordnung}

Die Einberufung der Mitgliederversammlung erfolgt in Textform durch den
Vorstand mit einer Frist von mindestens zwei Wochen. Hierbei ist die
Tagesordnung bekanntzugeben und die nötigen Informationen zugänglich zu
machen. Der Fristablauf beginnt mit dem auf die Absendung des
Einladungsschreibens folgenden Tag. Das Einladungsschreiben gilt dem Mitglied
als zugegangen, wenn es an die dem Vertretungsvorstand zuletzt bekannt
gegebene Anschrift gerichtet wurde.

Anträge zur Tagesordnung sind mindestens drei Tage vor der
Mitgliederversammlung beim Vorstand einzureichen. Über die Behandlung von
Initiativanträgen entscheidet die Mitgliederversammlung.

\SubParagraph{title=Versammlungsleitung}

Die Mitgliederversammlung wird vom Vorstandsvorsitzenden, bei dessen
Verhinderung von dem/der stellvertretenden Vorsitzenden, bei dessen/deren Verhinderung
von einem anderen Mitglied des Vorstandes geleitet. Ist kein Vorstandsmitglied
anwesend, bestimmt die Mitgliederversammlung den Versammlungsleiter.

Der Protokollführer wird vom Versammlungsleiter bestimmt.

\SubParagraph{title=Beschlussfähigkeit und Mehrheitserfordernisse}

Jede ordnungsgemäß einberufene Mitgliederversammlung ist beschlußfähig.

Beschlüsse werden mit der einfachen Mehrheit der anwesenden Mitglieder gefasst,
soweit die Satzung für den Einzelfall keine andere Regelung vorsieht.

Ein Beschluss über Satzungsänderungen oder über die Auflösung des Vereins ist
nur rechtswirksam, wenn er in der Einladung wörtlich angekündigt wurde,
mindestens drei stimmberechtigte Mitglieder an der Abstimmung teilnehmen und
der Beschluss mit Dreiviertelmehrheit gefasst wird.\label{MV-Aufloesung}

\SubParagraph{title=Abstimmungen und Wahlen}

Jedes aktive Mitglied hat eine Stimme. Stimmübertragungen sind nicht
zulässig.\label{Stimmrechte}

Wenn ein Drittel der stimmberechtigten Mitglieder ein anderes
Wahlverfahren verlangt, bestimmt der Versammlungsleiter die Art der
Abstimmung.

Für die Dauer der Durchführung von Vorstandswahlen wählt die
Mitgliederversammlung einen Wahlausschuß.

Jedes aktive Vereinsmitglied kann sich oder ein anderes aktives Mitglied zur
Wahl vorschlagen. Das nominierte Mitglied muß die Kandidatur akzeptieren, um
zur Wahl zugelassen zu werden.

Die Mitglieder des Vorstandes werden einzeln gewählt, zuerst der/die
Vorstandsvorsitzende, dann der/die stellvertretende Vorsitzende und zuletzt
die übrigen Mitglieder.

Es gilt der/die Kandidat/-in als gewählt, der mehr als die Hälfte der
abgegebenen gültigen Stimmen erhalten hat und die Wahl annimmt. Ist die
einfache Mehrheit nicht erreicht worden, findet im zweiten Wahlgang eine
Stichwahl zwischen den beiden Kandidaten statt, die die meisten Stimmen
erhalten haben.

Kann eine Stichwahl nicht stattfinden, z.B.\ weil ein/-e Kandidat/-in die Wahl
nicht annimmt, erfolgt umgehend eine Neuwahl aus den übri\-gen
Kandidaten/-innen. Stehen keine übri\-gen Kandidaten/-innen zur Wahl, wird das
Amt kommissarisch vom/von der Vorgänger/-in weitergeführt, bis ein/-e neue/-r
Kandidat/-in gewählt wurde, ggf.\ auf einer außerordentlichen
Mitgliederversammlung.

Bei Stimmengleichheit entscheidet die Stimme des/der Vorsitzenden bzw.\ bei
dessen/deren Verhinderung die Stimmen des/der stellvertretenden Vorsitzenden. Bei
trotzdem gegebener Stimmgleichheit zweier Kandidaten während Vorstandswahlen
entscheidet der Versammlungsleiter durch Ziehung eines Loses.

\SubParagraph{title=Niederschriften}

Über die Beschlüsse der Mitgliederversammlung ist ein Protokoll anzufertigen,
welches vom/von der Versammlungsleiter/-in und dem/der Protokollführer/-in zu
unterzeichnen ist und folgende Informationen enhalten muß:
\begin{compactenum}[\hspace{2em}1.]
  \item Ort und Zeit der Versammlung,
  \item Name des/der Versammlungsleiters/-in und des/der Protokollführers/-in,
  \item Anzahl und Namen der erschienenen Mitglieder,
  \item Feststellung der ordnungsgemäßen Einberufung und Beschlussfähigkeit,
  \item die Tagesordnung,
  \item die gestellten Anträge, das Abstimmungsergebnis (Zahl der Ja-Stimmen,
    Nein-Stimmen, Enthaltungen und ungültigen Stimmen), die Art der
    Abstimmung,
  \item Satzungs- und Zweckänderungsanträge,
  \item Beschlüsse, die wörtlich aufzunehmen sind.
\end{compactenum}

Das Protokoll ist binnen zwei Wochen vereinsöffentlich zur Verfügung zu
stellen.

\SubParagraph{title=Beschlüsse außerhalb ordentlicher Versammlungen}\label{AusserordentlicheBeschluesse}

Die Mitgliederversammlung kann außerhalb von Sitzungen Beschlüsse in
elektronischer Form (§~126a BGB) fassen, wenn alle Mitglieder mit einer
Vorlaufzeit von zwei Wochen per E-Mail davon in Kenntnis gesetzt
sind.\label{eBeschluss}

Bekanntzugeben ist der Beschlussvorschlag mit Erläuterungen sowie der
Bestimmung, in welcher Form und Frist die Stimmen abzugeben sind. Nach Ablauf
der Frist eingehende Stimmen werden nicht mehr berücksichtigt.

Die Abstimmung erfolgt per E-Mail an die E-Mail-Adresse des/der
Vorstandsvorsitzenden bzw.\ des/der von ihm/ihr eingesetzten Vertreters/-in,
oder an eine in der Auf\/forderung abweichend genannte Adresse.

Über die Abstimmung ist unter Nennung des Beschlusstextes und der abgegebenen
Stimmen vom Vorstandsvorsitzenden bzw.\ von einem eingesetzten Vertreter
Protokoll zu führen. Das Protokoll wird vereinsöffentlich zugänglich gemacht
und den Mitgliedern per E-Mail angezeigt.

Geben alle Mitglieder Ihre Stimme bereits vor Ablauf der zweiwöchigen
An\-kün\-di\-gungs\-frist gem.~\refParS{eBeschluss} ab und erklären sich
zusammen mit der Stimmabgabe zur vorzeitigen Beschlussfassung bereit, gilt der
Beschluss entsprechend der Stimmenmehrheit als gefasst bzw. nicht gefasst, sobald
die letzte Stimme eingegangen ist. Auf die vorfristige Entscheidung ist in der
Niederschrift hinzuweisen.

Im Übrigen wird das Verfahren durch Vorstandsbeschluß festgelegt.

\Paragraph{title=Der Vorstand}

Der Vorstand besteht aus mindestens fünf und nicht mehr als neun Mitgliedern:
\begin{compactenum}[\hspace{2em}1.]
  \item dem/der Vorsitzenden,
  \item dem/der stellvertretenden Vorsitzenden,
  \item dem/der Schriftführer/-in,
  \item dem/der Kassenwart/-wärtin,
  \item sowie mindestens einem und maximal fünf Beisitzern.
\end{compactenum}

Sind drei oder mehr Vorstandsmitglieder dauernd an der Ausübung ihres Amtes
gehindert, so sind unverzüglich Nachwahlen anzuberaumen.

Die Amtsdauer der Vorstandsmitglieder beträgt ein Jahr. Die Wiederwahl ist
zulässig. Der jeweils amtierende Vorstand bleibt bis zu dem Tag im Amt, an dem
ein neu gewählter Vorstand die Amtsgeschäfte übernimmt.

Der Vorstand beschließt über alle Vereinsangelegenheiten, soweit sie nicht
eines Beschlusses der Mitgliederversammlung bedürfen. Er führt die Beschlüsse
der Mitgliederversammlung aus.

Beschlüsse des Vorstandes müssen mit einfacher Mehrheit gefasst werden. Der
Vorstand ist beschlußfähig, wenn mindestens drei Mitglieder anwesend sind.

\SubParagraph{title=Vertretung}

Der vertretungsberechtigte Vorstand im Sinne des § 26 BGB besteht aus dem/der
Vorsitzenden und dem/der stellvertretenden Vorsitzenden.

Der/die Vorsitzende des Vorstands und sein/-e Stellvertreter/-in können den
Verein gerichtlich und außergerichtlich einzeln vertreten. Für Rechtsgeschäfte
mit einem finanziellen Volumen von über \EUR{5.000}, sowie der Einstellung und
Entlassung von Angestellten und der Aufnahme von Krediten ist nur eine
gemeinschaftliche Vertretung zulässig.

Der/die Kassenwart/-wärtin verwaltet die Vereinskasse und führt Buch über die
Einnahmen und Ausgaben des Vereins. Gegenüber Banken erhält er/sie hierfür
entsprechende Handlungsvollmachten. Er/sie ist berechtigt, Zahlungsanweisungen
entsprechend den Beschlüssen des Vorstandes oder der Mitgliederversammlung zu
unterzeichnen.

\SubParagraph{title=Vorstandssitzungen und -beschlüsse}

Der Vorstand trifft sich zu regelmäßigen Sitzungen, wenn die Geschäfte es
notwendig machen oder zwei Drittel der Vorstandsmitglieder die Einberufung
einer Vorstandssitzung fordern. Vereinsfremde Personen können zur Teilnahme
an Vorstandssitzungen ohne Stimmrecht zugelassen werden.

Hinsichtlich etwaiger Vorstandsbeschlüsse außerhalb ordentlicher Vorstandssitzungen
gilt \refParagraph{AusserordentlicheBeschluesse} sinngemäß. Die Einspruchsfrist
beträgt bei Vorstandsbeschlüssen außerhalb ordentlicher Vorstandssitzungen jedoch
abweichend von \refParagraph{AusserordentlicheBeschluesse}~\refParS{eBeschluss} für
Vorstandsmitglieder fünf Tage.

Sämtliche Vorstandssitzungen sowie sämtliche Beschlussfassungen durch
den Vorstand sind vereinsöffentlich.

Über sämtliche Vorstandssitzungen und Beschlüsse, insb. auch über jene
Beschlüsse, die außerhalb regulärer Vorstandssitzungen getroffen werden, hat
der Vorstand Protokolle zu führen. Die Protokolle sind innerhalb von höchstens
zwei Wochen ab Datum der Sitzung oder des Beschlusses vereinsöffentlich
zu\-gäng\-lich zu machen.

\SubParagraph{title=Satzungsänderungsvollmacht}

Satzungsänderungen, die von Aufsichts-, Gerichts- oder Finanzbehörden verlangt
werden, kann der Vorstand mit Zweidrittelmehrheit von sich aus vornehmen.
Diese Satzungsänderungen müssen der nächsten Mitgliederversammlung mitgeteilt
werden.

\SubParagraph{title=Vergütung und Versicherung}

Kein Vorstandsmitglied darf ein bezahltes Beschäftigungsverhältnis innerhalb
des Vereins wahrnehmen.

Die Vorstandstätigkeit wird nicht vergütet, erfolgt also ehrenamtlich.

% \pagebreak\enlargethispage{\textheight}

Der Vorstand ist berechtigt, nach eigenem Ermessen insbesondere folgende
Versicherungen auf Kosten des Vereins abzuschließen:
\begin{compactenum}[\hspace{2em}1.]
  \item Vereinshaftpflicht,
  \item Vermögensschadenhaftpflicht,
  \item Veranstalter-Haftpflichtversicherung,
  \item Dienstreiserahmenversicherung.
\end{compactenum}

\Paragraph{title=Die Ressorts}

Zu besonderen Themen und zur Erledigung spezieller Aufgaben richtet der
Vorstand Ressorts ein.

Die Einrichtung eines Ressorts kann jedes aktive Mitglied anregen.

Jedes Ressort wird von einem aktiven Vereinsmitglied geleitet, welches vom
Vorstand eingesetzt wird.

Der/die Leiter/-in eines Ressorts darf auch Nichtmitglieder in das Ressort
berufen, wenn es der Zweckerfüllung dient.

Die Ressorts geben sich bei Bedarf ihre Arbeitsrichtlinien selbst. Die
Arbeitsrichtlinien dürfen nicht der Satzung des Vereins widersprechen.

Der/die Leiter/-in jedes Ressorts erstattet der Mitgliederversammlung und nach
Auf\/\-for\-derung dem Vorstand Bericht über die Aktivitäten des Ressorts.

Auf Initiative des/der Ressortleiters/-in kann der Vorstand einem Mitglied
eines Ressorts per schriftlicher Vollmacht eine beschränkte
Vertretungsbefugnis nach außen gewähren, sofern dieses für die Tätigkeiten des
Ressorts erforderlich ist.

Ein Ressort wird geschlossen, wenn die Aufgaben für die es eingerichtet wurde
erledigt sind, oder wenn der Vorstand dies beschließt. Dies ist mit einem
Endbericht des Ressortleiters bzw.\ des Vorstands zu protokollieren. Etwaige
Vollmachten sind sofort zurückzugeben.

\Paragraph{title=Haftung des Vereins}

Der Verein und seine Organe und Erfüllungs- und Verrichtungsgehilfen haften
nicht für die vom Verein angebotenen Dienste und Informationen sowie deren
Folgen, und zwar weder für die Richtigkeit noch Vollständigkeit, noch daß sie
frei von Rechten Dritter sind oder der Nutzer rechtmäßig handelt, indem er
Daten zugänglich macht, anbietet oder übermittelt.

Für Schäden, die daraus entstehen, daß die Dienste und Informationen des
Vereins nicht oder nur eingeschränkt nutzbar sind, übernehmen der Verein und
seine Organe und Erfüllungs- und Verrichtungsgehilfen weder gesetzliche noch
vertragliche Haftung.

Dieser Haftungsauschluß gilt nicht bei Vorliegen von Vorsatz oder grober
Fahr\-läss\-ig\-keit.

\end{contract}

\pagebreak\section*{Unterschriften der Gründungsmitglieder}

\vspace{3em}

\rule{6cm}{0.4pt} \\
\noindent Martin Gerhard Loschwitz \\
\vspace{2em}

\rule{6cm}{0.4pt} \\
\noindent N.N. \\
\vspace{2em}

\rule{6cm}{0.4pt} \\
\noindent N.N. \\
\vspace{2em}

\rule{6cm}{0.4pt} \\
\noindent N.N. \\
\vspace{2em}

\rule{6cm}{0.4pt} \\
\noindent N.N. \\
\vspace{2em}

\rule{6cm}{0.4pt} \\
\noindent N.N. \\
\vspace{2em}

\rule{6cm}{0.4pt} \\
\noindent N.N.

\end{document}
